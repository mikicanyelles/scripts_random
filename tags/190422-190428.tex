\chapter{22/04/19-28/04/19 week}

\section{23-24/04/19}\label{day:190424}
\tags{scan, Seminario Method, 5NI6, MCPB.py, AMBER}
\Hydroxilation

	It has been sent again the calculation of the $r_1$ scan from the optimised structure of the frame 128 (Table \ref{calctab:309021}). The previous calculation didn't work correctly because of problems on the cluster (or this seems to be the reason).
	
	\CalcTable{scan\_r1\_128\_nomicro}{309021}{\tStatusDoing}{\tPicard}{borg2-32}{$r_1$ scan from optimised structure from frame 128 of the QM/MM MD trajectory. 6-31G* basis set used for all the atoms. $r_c$ is the distance between Zn and epoxide and the $r_s$ (distance step) is $-0,1$ \AA.}{}

	On the other hand, parameters for the MM MD of are being obtained following \href{http://ambermd.org/tutorials/advanced/tutorial20/mcpbpy.htm}{the Seminario Method} from the AMBER tutorials webpage. Because of lots of different problems with the Gaussian09 calculations (which are not explained in this notebook), the generation of parameters has been restarted. The first step of \texttt{MCPB.py} is being carried out (Tables \calref{309154}, \calref{309155}). 

	\CalcTable{5NI6\_small\_opt.com}{309154}{\tStatusDone}{\tPicard}{borg1-02}{Optimisation and frequencies calculations of the small model of the coordination sphere of Zn for 5NI6 + epoxy-DHA Michaelis complex}{}
	\ \\
	\CalcTable{5NI6\_large\_mk.com}{309155}{\tStatusDone}{\tPicard}{borg1-04}{Optimisation and RESP calculations of the large model of the coordination sphere of Zn for 5NI6 + epoxy-DHA Michaelis complex}{}
	


\section{25/04/19}\label{day:190425}
\tags{5NI6, MCPB.py, Seminario Method}

\Tasks

	\textcolor{darkgray}{
		\begin{todolist}
			\item[\done] Check scan \calref{309021}
			\item[\done] Check freqs \calref{309154}
			\item Generate a colour legend for task's priorities
			\item Look for \emph{comments} package.
			\item Add \emph{mu2018} article to bibliography (so add bibliography)
		\end{todolist}
	}

\Hydroxilation
	
	The scan calculation \calref{309021} is still calculating and the first point hasn't been reached yet. Two (RMS step and RMS grad) of the three parameters to be converged have been converged already, and the other three are not really far from being converged, so maybe tomorrow the first point of the scan calculation is achieved.
	
	On the other hand, and regarding the generation of parameters of LTA$_4$H enzyme (5NI6) + epoxy-DHA Michaelis complex, the RESP calculation of the large system has already been completed (\calref{309155}). Moreover, the optimisation of the small system has also been completed (\calref{309154}), while the calculation of frequencies is still ongoing but it is expected to be completed during the night.
	
	

\section{26/04/19}\label{day:190426}

\Tasks

	\textcolor{darkgray}{
	\begin{todolist}
		\item[\done] Generate parameters for 5NI6 + epoxy-DHA (complete steps 2, 3 and 4 and tleap execution from the Seminario Method).
		\item[\done] Send MD of 5NI6.
		\item Generate a colour legend for task's priorities
		\item Look for \emph{comments} package.
		\item Add \emph{mu2018} article to bibliography (so add bibliography)
		\item Create a list of tags (with hyperlinks if possible) and tag all the work done each day by program, type of calc, system, etc.
	\end{todolist}
	}
	
	
\Hydroxilation
\tags{MCPB.py, tleap, Seminario Method, fixed, MM MD, LTA4H, 5NI6, parmed, LJ parameters, prmtop edition}

	\subsubsection{MD 5NI6 + epoxy-DHA Michaelis Complex}
		Calculations \calref{309154} and \calref{309155} have finished correctly, so the further steps of the Seminario Method can be completed. Step 2, 3 and 4 have been completed without any problem and the \texttt{frcmod} file of the Zn and its environment has been created. The input file for \tleap{} has been created, but the force field of the protein has been changed from \progkeys{source oldff/leaprc.ff14SB} to \progkeys{source leaprc.protein.ff14SB} and \progkeys{source leaprc.water.tip3p} has been added before the building of the water box.
		
		During the execution of \tleap{}, two errors besides the one of the TIP3P water's parameters have appeard. Both were related to two different residues (from the \files{pdb}): the ALC and the LYN. It has been found that the ALC residue is in fact an ALA residue and that the LYN was, in fact, a protonated LYS (so a LYS instead a LYN). Both residue names have been changed and both problems have been fixed.
		
		Once the final \files{prmtop} and \files{inpcrd} files are obtained, the trajectory is computed. It has the main 5 stages: minimisation, heating, NPT equilibration, NVT equilibration and 100~ns of production.
	
		%%%% EXPLAIN IN MORE DETAIL THE MD!!!
		
		\CalcTable{MM MD of 2MT 5NI6 with epoxy-DHA}{MD-5NI6-EDH-1}{\tStatusError}{Local (Taxman)}{}{}{}
		
		On the second stage, the heating, the calculation has reached an error which has occured during the second minimisation step, where one of th H of the WT1 (the coordinated water) collides with the hydrogen of the GLH294. The protonation state of the GLH294 has been compared to the protonation state of the GLU296 of the 2VJ8 structure of LTA$_4$H and it has been found that in this structure it wasn't protonated, so it has been manually deprotonated and the parameters have been generated again. With this new parameters, the trajectory has been resent (\calref{MD-5NI6-EDH-2}).
		
		\CalcTable{MM MD of 2MT 5NI6 with epoxy-DHA}{MD-5NI6-EDH-2}{\tStatusError}{Local (Taxman)}{}{GLH294 has been manually deprotonated}{MD-5NI6-EDH-1}
		
		In this new trajectory another error has been obtained. In this case, the H of the coordinated water collides with the epoxide, so Lennard-Jones parameters have to be added. 
		
		In order to added these parameters, the topology and parameters file (the \files{prmtop} file) will be modified using \parmed. A well depth of \kcalmol{0.0047} and a minimum radius of 0.3019~\AA{} will be added to both H of WT1 (the one coordinated to Zn) in the following way:
		
		\terminalblock{
			> pye; parmed -p 5NI6\_GLU294\_solv.prmtop\\
			>> changeLJSingleType @9733 0.3019 0.0047\\	
			>> changeLJSingleType @9734 0.3019 0.0047\\	
			>> outParm 5NI6\_GLU294\_solv\_VdW.prmtop
		}
		
		With this edited topology and parameters file, the MM MD is sent again. The first attempt fails for no reason, but the second one works fine (\calref{MD-5NI6-EDH-3}).
		\begin{updates}
			\emph{28/04/19 Update: job has failed because production is being carried out in the \files{prod/} subdirectory but the \files{4\_eq\_5.rst} file (the structure from the last frame of the last equilibration step) wasn't copied. It has been resent and it is spected to be completed on 01/05/19, 08:00.}
		\end{updates}
		
		\CalcTable{MM MD of 2MT 5NI6 with epoxy-DHA}{MD-5NI6-EDH-3}{\tStatusDoing}{Local (Taxman)}{}{GLH294 has been manually deprotonated and LJ parameters for WT1 have been added}{MD-5NI6-EDH-2}
		
		
	\subsubsection{Optimisation of frame 128 without ECP}
		\emph{Summary: in order to test which combination of schemes and bases is the best, different calculations have been/are being carried out using the microiterative scheme or not and using ECP or not. In this case, the use of the microiterative scheme is tested.}
		
		\paragraph{Without microiterative scheme:}
			The optimisation has reached 100 cycles of optimisation because of the default limitation of DL-FIND but it has not been optimised. This wasn't unknown until now and the obtained structure was used on \calref{309021}, so it is being used a non-optimised reactant. This is not an error itself, but it requires the scan calculation to optimise first the reactant (with the bond restraint) and, then, start moving the atoms, so the reactant will be optimised but, to consider it a proper reactant it will be necessary to reoptimise without the bond restraint. Moreover, this will lead to a very time-expensive calculation.
			\begin{updates}
				\emph{On 29/04/19 it has taken $\approx140~\textrm{h}$ and the reactant is not obtained.}
			\end{updates}
			
		\paragraph{With microiterative scheme:}
			The microiterative scheme is used in order to facilitate the optimisation at a logarithmic level, but in this case it is still not possible to compare because the calculation without the microiterative scheme has not finished.
			
			In order to complete the optimisation, 1316 cycles have been necessary, what seems a lot of cycles. This can be because an effect of this concrete frame or it may be a result of using the DFTB level of theory for calculating, which may not be an appropriate level of theory for geometry optimisations. This will be tested when other frames optimised.
		
		
		
		
		
	
		
	


	
	
	
	
	
	
	
	
	
	
	
	
	
	
	
	
	
	
	
	
	
	
	
	
	
	
	
	
	
	
	
		
		
